


\documentstyle{article}

\topmargin -0.5in
\textheight 9in

\title{Quiz \#7}
\date{15 July 1992}
\author{Time: 10 minutes}
\begin{document}

\maketitle

0. What's your name?

\vspace{.5in}

Professor Spuyten, a noted eccentric, doesn't like to use the {\tt ++}
operator because he says reminds him of eyes looking at him.  He has
written the following program:

\begin{flushleft}
\verb% int main(void) % \\*
\verb% {% \\*
\verb%   int foo = 52;% \\*
\verb% % \\*
\verb%   increment(foo);% \\*
\verb%   /*% $foo$ should be 53 by now. \verb% */% \*
\verb!   printf("The value of foo is now %d.\n", foo);! \\*
\verb% % \\*
\verb% }% \\*
\verb% % \\*
\verb% void increment(int n) % \\*
\verb% {% \\*
\verb%   n += 1;% \\*
\verb%   return;% \\*
\verb% }%
\end{flushleft}

\vspace{8pt}

He explains: ``Since I don't like the increment operator, I wanted to
write a function to do the same thing.  I thought that if I did {\tt
increment(foo)}, the value of {\tt foo} would get bumped up to 53.  But
it didn't work.''

\vspace{8pt}
1.  What does Professor Spuyten's program print out, and why doesn't it
print 53?

\vspace{16pt}

2.  Correct Professor Spuyten's program so that it does what he wanted. 

\vspace{16pt}
3.  There are at least three other errors in Spuyten's program.  Find
them.

\vspace{16pt}

Answer on the other side of this sheet.
\end{document}
