
\documentstyle{article}

\topmargin -0.5in
\textheight 9in

\title{Quiz \#3}
\date{7 July 1992}
\author{Time: 10 minutes}
\begin{document}

\maketitle

0. What's your name?

\vspace{.5in}

1.  What are the values of {\tt x{\rm ,} y{\rm , and} z} when this code
is finished executing?

\verb% % \\*
\verb% int x, y, z; %  \\*
\verb%  %  \\*
\verb% x=12; y=9; z=23; %  \\*
\verb% x = y++ + ++z;      y = z / x; %  \\*
\verb% if ( y > x ) %  \\*
\verb%    z = 119;%  

\vspace{.5in}

2.  What are the values of {\tt x{\rm ,} y{\rm , and} z} when this code
is finished executing?  ({\bf Caution!} This is a trick question.)

\verb% % \\*
\verb% int x, y, z; %  \\*
\verb%  %  \\*
\verb% x=12; y=9; z=23; %  \\*
\verb% if ( x = y ) %  \\*
\verb%    z *= 2;%  

\vspace{.5in}

3.  What does this print?  And are are the values of {\tt x {\rm and} y}
when it is finished?

\verb% % \\*
\verb% int x, y%  \\*
\verb%  %  \\*
\verb% x=7; y=9; %  \\*
\verb% if ( --x > 6 &&  y++ > 8 ) printf("Foo.\n"); % \\*
\verb% else printf("Bar.\n"); %

\vspace{.5in}

4.  On the back of this sheet, write one sentence about each of three
things that the preprocessor does.


\end{document}

