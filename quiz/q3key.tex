
\documentstyle{article}

\def\brac#1{$<$#1$>$}
\def\Int{{\tt int}}
\def\int{\brac{\Int}}
\def\int{\brac{\Int}}

\topmargin -0.5in
\textheight 9in

\parskip 8pt

\title{Quiz \#3 Answers}
\date{8 July 1992}
\author{}
\begin{document}

{\bf 1.  What are the values of {\tt x{\rm ,} y{\rm , and} z} when this code
is finished executing?

\verb% % \\*
\verb% int x, y, z; %  \\*
\verb%  %  \\*
\verb% x=12; y=9; z=23; %  \\*
\verb% x = y++ + ++z;      y = z / x; %  \\*
\verb% if ( y > x ) %  \\*
\verb%    z = 119;%  
}

First, {\tt x} gets 12, {\tt y} gets 9, and {\tt z} gets 23.

Then we have to evaluate {\tt x = y++ + ++z;}.  The value of {\tt y++}
is 9, and the compiler remembers to bump up the value of {\tt y} by 1
sometime before the end of the statement.  The value of {\tt ++z} is 24,
which is what {\tt z} will be after the compiler bumps up the value of
{\tt z} by 1, which it must also do before the end of the statement.  So
{\tt x} gets $9+24$, which is 33.  That's the end of the statement, so
now {\tt y} and {\tt z} have both been bumped, and {\tt y} is 10 and
{\tt z} is 24.

The next statement is {\tt y = z / x;}.  {\tt z} is 24 and {\tt x} is
33, and they're both \int s, so {\tt /} means integer division, which
means we discard the fractional part of the result.  So {\tt y} gets 0.

Now {\tt y > x} is false, so the computer skips the whole {\tt if}
statement.  The final result is:  {\tt x{\rm : 33,} y{\rm : 0,} z{\rm :
24}}.  These were worth a point each.

{\bf 2.  What are the values of {\tt x{\rm ,} y{\rm , and} z} when this code
is finished executing?  ({\bf Caution!} This is a trick question.)

\verb% % \\*
\verb% int x, y, z; %  \\*
\verb%  %  \\*
\verb% x=12; y=9; z=23; %  \\*
\verb% if ( x = y ) %  \\*
\verb%    z *= 2;%  
}

This is a trick question because the condition in the {\tt if} clause is
{\tt x = y}, and {\em not}\/ {\tt x == y}.  The expression {\tt x == y}
compares {\tt x} and {\tt y}, yielding  true if they're equal and false
otherwise.  The expression {\tt x = y}, on the other hand, assigns the
value of {\tt y} to the variable {\tt x}, and its value is whatever
value got assigned to {\tt x}---in this case, 9.  So after the condition
in the {\tt if} clause is evaluated,
{\tt x} is 9, {\tt y} is 9, and {\tt z} is 23.  

Now, a condition is `true' when its value is not zero, and the value of
this condition is 9.  Therefore the statement {\tt z *= 2;} is executed.
This assigns the value 46 to {\tt z}.  So the final score is: {\tt x{\rm
: 9,} y{\rm : 9,} z{\rm : 46}}.  {\tt y} was easy and was worth a point;
the values of the other two depended on you knowing realizing what the
{\tt =} was doing and were worth two points each.

{\bf 3.  What does this print?  And are are the values of {\tt x {\rm and} y}
when it is finished?

\verb% % \\*
\verb% int x, y%  \\*
\verb%  %  \\*
\verb% x=7; y=9; %  \\*
\verb% if ( --x > 6 &&  y++ > 8 ) printf("Foo.\n"); % \\*
\verb% else printf("Bar.\n"); %
}

The thing to remember here is that {\tt \&\&} short-circuits.  That
means that the compiler evaluates the left-hand part of the {\tt\&\&}
expression first, and only evaluates the right-hand part if it needs to.
The value of {\tt --x} is 6, and the compiler remembers that it must
decrement {\tt x} before the end of the statement.  Since {\tt 6 > 6} is
false, the compiler knows that whatever's on the right of {\tt\&\&} is
irrelevant---the whole expression will be false no matter what.  So it
never evaluates the {\tt y++ > 8} part, and in particular it doesn't
evaluate {\tt y++}, so {\tt y} never gets bumped.

The condition was false, so the computer jumps to the {\tt else} clause
and prints {\tt Bar.}.  {\tt x} got decremented by this time, so {\tt x}
is now 6.  {\tt y} never got incremented at all, so {\tt y} is still 9.

Getting the {\tt Bar.} and the value of {\tt x} were worth a point
apiece; to get the value of {\tt y} you had to remember that {\tt \&\&}
short-circuited and so that was worth two points.

{\bf 4.  On the back of this sheet, write one sentence about each of three
things that the preprocessor does.}

The three things I was looking for in particular were:  

\begin{itemize}
\item The preprocessor removes comments, which are any text between and
including the sequences {\tt /*} and {\tt */}.
\item The preprocessor substitutes the appropriate text for the names of
manifest constants that have been defined with the {\tt\#define}
directive.
\item The preprocessor incorporates  the text of files included with the
{\tt\#include} directive into the input seen by the compiler.
\end{itemize}

Most people got at least two of these.  They were worth a point each.

\end{document}

