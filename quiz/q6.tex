
\documentstyle{article}

\topmargin -0.5in
\textheight 9in

\title{Quiz \#6}
\date{9 July 1992}
\author{Time: 10 minutes}
\begin{document}

\maketitle

0. What's your name?

\vspace{.5in}

Here's a loop:

\begin{flushleft}
\verb% long int n; % \\*
\verb% % \\*
\verb% n = 2; % \\*
\verb% printf("Powers of 2 up to 1,000,000.\n"); % \\*
\verb% while (n < 1000000) { % \\*
\verb!   printf("%ld\n", n);! \\*
\verb%   n *= 2; % \\*
\verb% }% \\*
\end{flushleft}

It prints out all the powers of 2 between 2 and 1,000,000.

1.  Write an equivalent loop using {\tt do{\rm--}while} instead of {\tt
while}.  
\vspace{2in}

2.  Write an equivalent loop using {\tt for} instead of {\tt
while}.  
\vspace{2in}

3.  Which of the three loop constructs do you think is best for this
example?  Why?

\end{document}
