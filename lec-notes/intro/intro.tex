
\documentstyle{article}

\parskip 8pt


\title{CSE 110}
\date{1 July 1992}
\author{Mark-Jason Dominus}

\begin{document}

\maketitle

\section{Who Am I?}

My name is Mark-Jason Dominus.  Unlike most of the people who teach this
course, I am not a Computer Science graduate student.  Instead, I am a
systems programmer for the CIS department here at Penn; my undergraduate
work was in mathematics. 

\section {What to Call Me}

It is out of style these days to call people by their last names, but I
would prefer that you call me by mine.  `Mark-Jason' is awkward to say,
and `Mark' is too common.  Most people who know me call me `Dominus'.
If you feel uncomfortable with that, it is all right to call me `Mark'.
I do not have a doctorate, so do not call me `Doctor'.  I am not a
professor, so do not call me `Professor'.

\section{How You Can Reach Me}

    My office is in Moore~457, which is in a different building than the
one our classroom is in.  The buildings are attached.  To get to my
office from the classroom, go out the door, walk forwards a little, and
go down the long corridor on the left.  Now you are in the other
building.  Go to the fourth floor.  Do not try to go to the fourth floor
before you get to the other building, because this building doesn't have
a fourth floor.

    We will agree on office hours today, but I am usually in the office
between 10 AM and 7 PM, so I should be easy to find.  I might also be in
Moore~207, which is right around the stairwell from our classroom.
Of course you can also make an appointment to see me at a certain time.

    My work telephone number is 898-5617, but please don't call me
unless it is extremely urgent.  If you can send electronic mail, my
address is
\verb+mjd@cis.upenn.edu+.  You can send mail any time.

    I have a mail folder in the CIS business Office in Moore~557.
Room~557 is in the same other building as my office.

\section{What the Course is About}

The name of this course is {\em Introduction to Programming with C}\/.  So
the course is primarily about how to program, and general techniques of
programming which are independent of the language you are using.  Since
you can't program without a programing language, the course is also
about the C language itself, but this is secondary.  Since you can't
program without having a real computer and an operating system to
program on, the course is peripherally about microcomputers based on the
intel 80286 chip and about MS-DOS and Borland Turbo C++.

\section{Please Interrupt}

I don't like this business of students having to raise their hands in
class in order to ask questions---it seems to me too much like
kindergarden.  I don't like lectures; I would rather have a dialogue.
In a class as small as this one it's possible to do that.  We have a lot
to do and questions are important, and we can't waste time by having
everyone wait around with their arms sticking up in the air.  If you
want to ask a question or make a comment, you can raise your hand if you
want, but I would be happier if you would just interrupt me.  Similarly,
if I am going too fast or too slow, or if you want me to repeat
something, or if you are getting bored, also interrupt.

I am supposed to be serving you, and not the other way around.

\end{document}


