%
% CSE 110 Lecture Notes
%
% Entire contents are copyright 1992 by Mark-Jason Dominus.
% All rights reserved.  Unauthorized reproduction prohobited.
%

\section{More Operators}

    C has many operators.  Some of them, like {\tt +}, are {\em binary},
which means that they require two operands, as in {\tt 4 + 5}.  Others
are {\em unary}, which means they require only one operand.  We'll see
an example of this in section \ref{unary-op}.

\subsection{Arithmetic}

    Arithmetic operators include {\tt +}, {\tt -}, {\tt *}, and {\tt /}
for addition, subtraction, multiplication, and division.  Division is a
little odd: Its semantics change depending on the types of its operands.
If both operands are \int s, then {\tt /} represents {\em integer
division}, in which the fractional part of the result is discarded.  For
example, the value of the expression {\tt 13/5} is 2.

    {\tt \%} denotes the {\em modulus}\/ operator: If {\em a}\/ and {\em
b}\/ are integer expressions, then {\em a}\/{\tt \%}{\em b}\/ is the
remainder when {\em a}\/ is divided by {\em b}\/.  For example, the
value of the expression \mbox{\tt 13 \% 5} is {\tt 3}, because 3 is the
remainder when you divide 13 by 5.  If one of the operands is an
expression whose value isn't of integer type, that's an error and the
compiler won't compile the program.

\subsection{Assignment}
\label{assignment}

    {\tt +=} is an assignment operator.  Like {\tt =}, its left operand
must be an lvalue.  \mbox{\tt x += 2 } means the same thing as \mbox{\tt
x = x + 2}.  It's more natural to think ``Add 2 to {\tt x}'' and to
write \mbox{\tt x += 2;} than it is to think ``Get {\tt x}, add 2, and
put it back.'' and write \mbox{\tt x = x + 2;}.  Furthermore, in an
expression like

\verb% yyval[yypv[p3+p4] + yypv[p1+p2]] += 2 %

\noindent the assignment operator makes the code easier to understand,
since the reader doesn't have to check painstakingly that two long
expressions are the same, or to wonder why they're not.\footnote{I
swiped this example from {\em The C Programming Language}\/, by
Kernighan and Ritchie.}

Similarly, there are {\tt -=}, {\tt *=}, {\tt /=}, {\tt \%=} operators,
all defined analogously.

\subsection{Increment and Decrement}

    Adding one to something is such a common operation that there's a
special name for it and yet another notation for it.  Adding 1 to a
quantity is called {\em incrementing} it.  There's a special increment
operator, {\tt ++}.  If we write {\tt x++} or {\tt ++x} in an
expression, then, sometime before the computer executes the next
statement, it will add 1 to the value stored at {\tt x}.\footnote{Of
course, when we say ``the value stored at {\tt x},'' we are implying
that {\tt x} is an lvalue---you can't do {\tt 4++} to increment the
value of 4; the compiler will complain.}

    The values of {\tt x++} and {\tt ++x} differ, however: if {\tt x} is
12, then the value of {\tt x++} is 12 and the value of {\tt ++x} is 13.
The notion is that if you use {\tt ++x}, the compiler increments {\tt x}
before getting its value, and if you use {\tt x++}, the compiler
increments {\tt x} after geting its value.  So if the value of {\tt x}
is 119, then after

\verb%  y = x++ ; %

\noindent {\tt y} would be 119 and {\tt x} would be 120, but after

\verb%  y = ++x ; %

\noindent {\tt y} would be 120 and {\tt x} would be 120.

When the compiler actually chooses to do the increment is None of Your
Business, as long as the increment happens before the next
statement.\footnote{Actually it has to happen before the next {\em
sequence point}\/.  We know about two kinds of sequence points:
semicolons are sequence points, and also there is a sequence point just
before every function call.}

Question:  What happens if {\tt x} is 119, and we do:

\verb% y = x++ + x++ ; %

\noindent ?  Depending on when the incrementing actually happens, {\tt y}
might get 238, or 239, right?  Wrong.  For various technical reasons
the standard says that if you try to modify the same object twice in one
statement,\footnote{Really it says twice without a sequence point in
between.} you get undefined behavior---the compiler is allowed to do
whatever it likes, including nothing, printing a warning, erasing
itself, or teleporting elephants into the room.  Similarly

\verb% x = x++ ; %

\noindent is undefined.  On the other hand, if {\tt x} is 2, then

\verb% y = x * x++ ; %

\noindent is perfectly legal, ({\tt x} and {\tt y} are each modified
only once before the end of the statement) and might assign {\tt y} the
value 4 or the value 6, depending on whether the compiler does the
increment before or after it evaluates the first {\tt x}.  In general
it's best to avoid such situations.

There is a {\tt -\relax-} operator, which is just like {\tt ++}, except
that it subtracts 1 instead of adding 1.  This is called {\em
decrementing}\/.

\subsection{Precedence}

Consider this expression:

\verb% 2 * 3 + 4 %

\noindent does the compiler do {\tt 2 * 3} first and then add {\tt 4}, yielding
10?  Or does it do {\tt 3 + 4} first and then multiply by {\tt 2},
yielding 14?

The rules are consistent with regular mathematics: Multiplication and
division (meaning {\tt *}, {\tt /}, and {\tt \%}) happen before {\tt +}
and {\tt -}.  So the example above evaluates to 10.

Assignment happens way late, after almost everything else, because if
you write

\verb% x = y + 4 ; %

\noindent you never ever mean that you want to store the value of {\tt
y} in {\tt x} and then add 4 to that result---you always mean that you
want to add 4 to the value in {\tt y} and then store that result into
{\tt x}.

As in mathematics, expressions in parentheses ({\tt (} and {\tt )}) get
evaluated first.  So the value of

\verb% 2 * ( 3 + 4 )  %

\noindent is 14.

{\tt ++} and {\tt --} have higher precedence than most other things,
including everything we've seen so far.  That's so that the {\tt ++} in
\mbox{\tt y * x++} applies just to the {\tt x} and not to the {\tt y * x}.  

\section{The Preprocessor}

Before the compiler starts in on its work in earnest, it transforms your
program a little.  On some systems this transformation is done by a
separate program, called a {\em preprocessor}\/, but in Turbo C++, the
preprocessor is built into the compiler.  There are three important
transformations and a few unimportant ones that we won't discuss.

\subsection{Macros}

If your source code contains a line like

\verb%#define PI 3.141592654 %

\noindent the preprocessor defines a {\em macro}.  What this means is that from
now on, every time it sees the symbol {\tt PI}, it will replace it with
the sequence {\tt 3.141592654}.  The compiler proper will never find out
about {\tt PI} at all; as far as it's concerned, you wrote out {\tt
3.141592654} in full every time.

You can use this for three things:

\begin{itemize}
\item You can use it to clarify the code, by writing things like {\tt
PI}, rather than {\tt 3.141592654} all over the place.

\item You can use it to set up {\em manifest constants} that might
change over time.  For example, suppose you are writing accounting
software for an insurance company; the medical insurance deductible is
\$200, and you want to compute the payment.  You could write

\verb% payment = claims - 200 ; %

\noindent but then if the deductible ever changed, you'd have to go and
find all the {\tt 200}'s in your code and change every one.  Worse, it
might be that not every {\tt 200} is actually a deductible---you'd have
to decide which ones to change.  It's much better to do

\verb%#define DEDUCTIBLE 200 %

\noindent and then you can write

\verb% payment = claims - DEDUCTIBLE ; %

\noindent and if you need to change the deductible, you just change it
in the {\tt\#define} directive and nowhere else.

\item You can use the macro facility to make your code into an
unreadable horror.  We will not see an example of how to do this.

\end{itemize}

Conventionally the names we give to macros contain only capital letters.

\subsection{Include Files}

If you write

\verb% #include <file.h> %

\noindent the compiler immediately pauses what it was doing, seeks out a
{\em header file} called {\tt file.h} in some standard
place,\footnote{Just what place this is is {\em implementation
defined}\/, which means that it's well-defined, and that it must be
documented, but it differs from system to system.} and pretends that the
entire contents of that file appeared in your source file in place of
the {\tt \#include} directive.  If the compiler doesn't find {\tt
file.h} in any of the standard places, it complaints.

One of the good things about C is that you can have a program whose
source lives in more than one file; then if you make changes to one file
you don't have to recompile all the others to make an executable.  But
if there's some information they need to share, you can put it in one
header file and have all of the source files {\tt \#include} it; again,
if the information changes, you only need to change the one copy in the
header file instead of going around and changing each source file.

Similarly, when you want to use a library function like {\tt printf},
there might be information you need to give to the compiler about that
function.  The people who wrote {\tt printf} can put whatever
information is necessary into a header file, and then you can include
the header file in your program before you use {\tt printf}.  In fact,
in order to use {\tt printf}, you have to {\tt\#include} the header file
{\tt stdio.h}, or else you get undefined behavior.\footnote{Our {\tt
square} program from section \ref{square-program} had undefined behavior
for this reason.  Fortunately, it did the right thing anyway.}

If you write {\tt\#include "file.h"} instead of {\tt\#include <file.h>},
the preprocessor looks for {\tt file.h} in the current directory before
it looks in the standard places.

Header files don't have to have a {\tt .h} extension, but they always
do.

\subsection{Comments}

The preprocessor provides a facility for including explanatory text,
called {\em comments}\/, into your program, without confusing the
compiler.  The rule is this: The sequences {\tt /*} and {\tt */} delimit
comments.  The preprocessor replaces the {\tt /*}, the {\tt */}, and
everything in between with blank space, which the compiler ignores.

It's too early for an enormous rant about the importance and proper
style of comments, but we'll have several later.

\section{Conditions}

If a program did the same thing every time we ran it, it wouldn't be
useful.  We have to have a way to perform certain actions only when
certain conditions are true.

\subsection{The {\tt if} Statement}

The {\tt if} statement has this form:

\begin{quote}
{\tt if (}{\it condition}\/{\tt )} {\it statement}

\end{quote}

The condition is just an expression.  We say that the condition is {\em
false}\/ if the value of the expression is zero, and we say that the
condition is {\em true}\/ otherwise.  To execute an {\tt if} statement,
the computer first evaluates the expression to decide if the condition
is true or not.  If the condition is true, the computer executes the
statement.  Otherwise, it doesn't.  

The statement could be a compound statement, or it could even be another
{\tt if} statement.

\subsection{Relational Operators}

C provides operators for comparing numbers.  The operator {\tt ==} tests
two expressions for equality, for example.  The expression {\tt a == b}
has the value 0 if {\tt a} and {\tt b} are not equal and {\tt 1} if they
are equal.  So we can write

\verb%  if ( a == b ) % \\*
\verb%      printf("a and b are equal\n"); %

If {\tt a} and {\tt b} are equal, the expression {\tt a == b} evaluates
to 1, so the condition is true, and the {\tt printf} statement gets
executed.  But otherwise, {\tt a == b} evaluates to 0, so the condition
is false and the {\tt printf} is not executed.

{\tt{\em a}\/ != {\em b}} is true when the value of expression {\em a}\/
is not equal to the value of expression {\em b}\/.  Similarly, we have
{\tt <}, {\tt >}, {\tt <=}, and {\tt >=}, for testing whether one
expression's value is less than another's, greater than another's, less
than or equal to another's, or greater than or equal to another's.

Relational operators have low precedence, just before assignments, but
after arithmetic.

\subsection{Boolean Operators}

Suppose we want the user to enter an integer between 1 and 10,
inclusive.  We want to write some code to print a rude message if the
user didn't do what we wanted.  Suppose the user's number is stored in
the variable {\tt x} Then we could write

\verb%% \\*
\verb% if ( x < 1 ) %                  \\*
\verb%   <print rude message> ;  %       \\*
\verb% if ( x > 10 ) %                  \\*
\verb%   <print rude message> ; %       

\noindent but then the code to print the rude message is the same both
times.  That's bad, because someday someone is going to change one and
not the other, and then the program will have different behavior where
it used to have the same behavior; or else someday both statements might
break\footnote{Maybe because the print function changed or something.}
and someone might fix only one of them by mistake.  A principal rule of
programming is to never ever have two pieces of code to do the same
thing.  Fortunately there's a better way to accomplish what we want:

\verb%% \\*
\verb% if ( x<1 || x>10 ) %                  \\*
\verb%   <print rude message> ; %       

{\tt ||} reads as `or', so we say ``if {\tt x} is less than 1 {\bf or}
{\tt x} is greater than 10...''.  {\tt ||} requires two operands, and an
expression with an {\tt ||} operator is true if either of its operands
are true, false if neither is true.

{\tt ||} has a special property: it {\em short-circuits}\/.  In the
example above, suppose the value of {\tt x} is 0.  The computer compares
{\tt x} with 1, and finds that {\tt x < 1} is true, and so we already
know that the rude message will be printed.  There's no longer any
reason to computer whether or not {\tt x > 10} is true; either way we'll
print the message.  And in fact in C when the computer is evaluating an
{\tt ||} expression, if the left-hand operand is true, then the computer
never evaluates the right-hand one at all.

Similarly, there's a {\tt \&\&} operator, which is pronounced `and'.  For
example:

\verb%% \\*
\verb% if ( x>=1 && x<=10 ) %                  \\*
\verb%   <print polite message> ; %       

\noindent ``If {\tt x} is greater than or equal to 1 {\bf and} {\tt x}
is less than or equal to 10, then print the polite message.''  An
expression with {\tt \&\&} is true if both its operands are true, false
otherwise.  {\tt \&\&} also short-circuits, so if {\tt x} is 0 in the
example above, then the computer will only bother to evaluate the {\tt x
>= 1} part above; since that part is false, it already knows that the
whole {\tt \&\&} expression is false, and there's no point in evaluating
the {\tt x <= 10} part.

\label{unary-op}
{\tt ||} and {\tt \&\&} are called {\em logical operators}\/ because
they operate on logical conditions such as {\tt x < 10}, rather than on
raw numbers like {\tt 12}.  There is one other logical operator: {\tt
!}, pronounced {\em not}\/.  {\tt !} is a unary operator; it takes only
one operand. When the computer wants to evaluate something like {\tt
!{\em x}\/}, it first evaluates {\em x}\/, and then if {\em x} \/ is
true, {\tt !{\em x}\/} is false, and vice-versa.

{\tt !} has higher precedence than anything else we've seen yet.  {\tt
\&\&} and {\tt ||} have lower precedence than anything except
assignment. {\tt \&\&} has higher precedence than {\tt ||}, which is
consistent with conventional mathematics.\footnote{There is a table of
operator precedence in your text, on pages
690--691.}\footnote{Incidentally, {\tt\&\&} and {\tt ||} are both
sequence points.}

