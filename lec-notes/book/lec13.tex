%
% CSE 110 Lecture Notes
%
% Entire contents are copyright 1992 by Mark-Jason Dominus.
% All rights reserved.  Unauthorized reproduction prohobited.
%

\section{Now You Know C}

We just spent a long time learning a lot of ins and outs of the C
language, and guess what?  We're almost done.  There's a very short list
of language features that we haven't already seen, and some of those we
won't bother with.  In other words:  You know the language.

This book is supposed to concern itself primarily with how to program,
and secondarily with the C language itself, and now we're finally
getting to studying programming instead of C.

Accordingly, we spent some time working on writing a simple program:
{\tt type} Complete code for {\tt type} appears in section
\ref{type-program}.

The point of working on this in class is to argue about style and
documentation, to think about issues of formatting, and most important,
program structure and how to write a good program.  

\section{{\tt type}, Episode 1}

The {\tt type} program is a fundamental MS-DOS utility.\footnote{It's a
fundamental utility in nearly all environments.  For example, under {\sc
UNIX}, {\tt type} is called {\tt cat}.}  When you run it, you give it
one or more filenames as arguments.  {\tt type} gets the contents of
each file in turn and writes that data to the standard output.  

This has at least uses:

\begin{itemize}
\item You can use {\tt type} to view a file by dumping its contents to
the screen.
\item You can use {\tt type} to concatenate\footnote{{\em concatenate}
means to stick things together in a row.  For example, if we concatenate
the words `foo' and `bar' we get the word `foobar'.} two files, by
redirecting the standard output of {\tt type} into a destination file.
\end{itemize}

\subsection{Notes About the Design}

We're writing this program with the {\em top-down design}\/ method,
which is one of very few effective techniques for writing a real
program.\footnote{Other techniques include: The {\em bottom-up}\/
approach, where you implement a few fundamental primitive functions, and
then some more complicated routines that depend on those, and then
finally build up the program from the building blocks you've written;
the {\em Mongolian hordes}\/ technique, where you just dive in and write
the first thing that comes into your head.  The Mongolian hordes
technique is not particularly effective.}  

Top-down design means that we start by thinking about what our program
has to do, in general terms:  ``It will have to open a file, read the
data from the file, write the data to the standard output, and close the
file.  It will have to do that over and over for each file named on the
command line.''  Some of the sentences in the description correspond
obviously with C constructs:  `repeat over and over' means a {\tt
while}. {\tt do{\rm--}while}, or {\tt for} loop.  Other sentences get
turned into functions, such as {\tt read\_data\_from\_file()}.

This description becomes our function {\tt main}.  Then we start writing
that functions that we called from {\tt main} that aren't written yet:
{\tt read\_data\_from\_file()}, for example.  If a function is
simple and we can see right away how to write it, we just go ahead; if
it seems more complicated, we can top-down design it in the same way.

Functions let us break a big problem into little problems.  Each
function solves a little problem; then all we have to do is put the
functions together.  If the little problems turn out to be too big to
handle, we can break them down into even littler problems.

\subsection{What we got Done Today}

\input type.tex

\subsection{Some Notes About the Design Process}

Our original plan was to have {\tt open\_file} handle errors and return
0 or 1 to {\tt main} to indicate success or failure.  But then we
realized that won't work---on success, {\tt open\_file} gets a {\tt FILE
*} from {\tt fopen}, and it {\em must}\/ return this {\tt FILE *} to
{\tt main} so that {\tt main} can pass it off to the other functions
that need it.  The function to read the data from the file will need
this value, as will the function that closes the file again.

So {\tt open\_file} must return a {\tt FILE *} on success.  The natural
value to have it return on failure is that same one that {\tt fopen}
itself returns on failure: {\tt NULL}.  Furthermore, {\tt open\_file}
isn't in a position to do any cleanup if {\tt fopen} fails; {\tt main}
will need to skip that file completely and not try to read it or close
it, but {\tt open\_file} can't do anything about that except warn {\tt
main} that something went wrong.  So {\tt open\_file} ended up being a
rather small function.  That's all right.  It's much easier to merge a
small function back into the place where it's called and eliminate the
call than it is to separate out a big block of code from one function
and put it into a new function.  Real programs often have one-line
functions. 

Our original plan was to have one function to read data in from the file
and one to write it out again.  But then we realized that since all we
ever do with the data we read in is write it out again right away, we
should put both things together and have one function ({\tt spew}) which
reads the data and writes it out immediately.  

I want to push very hard on the idea that designs and programs do not
spring forth out of your head fully-formed, like Athena from the
forehead of Zeus.  You get an idea, and then discover it doesn't work
right, and then you fix it, like we did here.  And if it doesn't work,
the you can try something else. 

\section{Miscellaneous Things We Discussed}

Actually writing a program brought up a few issues, some of them about
design and others about language features.

\subsection{Predefined Streams and the Standard Error Output}

When you run your program, three stream variables are already set up.
{\tt stdin} is a value of type {\tt FILE *}, which represents the
standard input; similarly {\tt stdout} represents the standard output.
Doing {\tt getc(stdin)} is {\em exactly}\/ the same as doing {\tt
getchar()}---in fact, {\tt getchar()} is usually a macro which expands
to {\tt getc(stdin)}.  Similarly {\tt printf} is usually a one-line
function which calls 

\begin{flushleft}
\verb! fprintf(stdout, ...); !
\end{flushleft}

\noindent to do the real work.

The third predefined stream is called {\tt stderr}, which is short for
{\em standard error output}\/.  It's pretty much equivalent to {\tt
stdout}---it's normally attached to the screen, so that things you write
to {\tt stderr} appear on the screen.  But it's not identical with {\tt
stdout}, and if the user redirects the standard output into a file by
putting {\tt > file} on the command line, the standard error is {\em
not}\/ redirected---it stays attached to the screen.

{\tt stderr} is there so that you have a place to write error messages.
If you used {\tt printf} to write your error messages, they'd go onto
the standard output, and might wind up in a file and never be seen.  But
if you write them with 

\begin{flushleft}
\verb! fprintf(stderr, "Error in line %d.\n", lineno); !
\end{flushleft}

\noindent instead, they go onto the screen no matter 
where the regular output is going, and the user can see them.

\subsection{{\tt continue}}

{\tt continue} is covered in section \ref{continue}.

It may seem like I keep pulling these things out of my hat, but really
we're near the end.  The only control structures we haven't seen yet are
{\tt switch}  and {\tt goto}.  We won't do {\tt goto}.  I kept thinking
I would bring up {\tt switch} when we needed it (like {\tt continue}),
but it just hasn't come up yet.

\subsection{Don't Ring the Bell}

If you ring the bell on the computer, everyone in the room hears it.

Therefore, the question you have to ask yourself before writing a
program that rings a bell to signal a certain event is, ``Is this event
so important that everyone in the room needs to know about it?''

Usually, the answer is `no'.

\subsection{Don't Worry About the Output Device}

One person said we should read the input and write the output in
80-character chunks, because the screen is probably 80 characters wide.

This is silly.  The input might be something like:

\begin{verbatim}
short
words
stars
groak
fubar
flesh
\end{verbatim}

\noindent in which case the fact that the screen is 80 characters wide is
completely irrelevant---the lines in the input are only 5 characters
each.  On the other hand, maybe the lines in the input are very long,
2000 or 300 characters.  

The specification for the program calls for ``Write the data to the
standard output.''

Whenever you start thinking about things like the screen width, you have
to step back and ask if the screen width is really relevant.  The way to
ask this is to say to yourself, `Does it make sense to run my program on
a terminal that prints the output on paper?'  In this case the answer is
`yes'.  Sure, you might want to type out the contents of some files,
even if you're typing out on paper.  Ask yourself `Would it make sense
to run my program if the screen were 1,065 characters wide?'  Sure, why
not?  We're just typing out files.

Even though {\em our}\/ displays are 80 characters wide, it's not
relevant to the program.  So don't put it in.
