%
% CSE 110 Lecture Notes
%
% Entire contents are copyright 1992 by Mark-Jason Dominus.
% All rights reserved.  Unauthorized reproduction prohobited.
%

\begin{flushleft}
\verb%#include <stdio.h>%
\\* \verb%#include <stdlib.h>%
\end{flushleft}

\begin{flushleft}
\verb%void hanoi(int num_rings, int start_peg, int end_peg);%
\\* \verb%void move(int ring_num, int start, int end);%
\end{flushleft}

\begin{flushleft}
\verb%int main(int argc, char **argv)%
\\* \verb%{%
\\* \verb%  int num_rings;%
\end{flushleft}

\begin{flushleft}
\verb%  if (argc != 2) {%
\\* \verb%    fprintf(stderr, "Usage: %\verb-%-\verb%s number_of_rings\n", argv[0]);%
\\* \verb%    return 1;%
\\* \verb%  }%
\end{flushleft}

\begin{flushleft}
\verb%  num_rings = atoi(argv[1]);%
\end{flushleft}

\begin{flushleft}
\verb%  hanoi(num_rings, 1, 3);%
\end{flushleft}

\begin{flushleft}
\verb%  return 0;%
\\* \verb%}%
\end{flushleft}

\begin{flushleft}
\verb%void%
\\* \verb%  hanoi(int num_rings,%
\\* \verb%        int start_peg, %
\\* \verb%        int end_peg)%
\\* \verb%{%
\\* \verb%  int spare_peg = 6 - start_peg - end_peg;%
\end{flushleft}

\begin{flushleft}
\verb%  if (num_rings > 0) {%
\\* \verb%    hanoi(num_rings - 1, start_peg, spare_peg);%
\\* \verb%    move(num_rings, start_peg, end_peg);%
\\* \verb%    hanoi(num_rings - 1, spare_peg, end_peg);%
\\* \verb%  }%
\end{flushleft}

\begin{flushleft}
\verb%  return;%
\\* \verb%}%
\end{flushleft}


\begin{flushleft}
\verb%void move(int ring_num, int start, int end)%
\\* \verb%{%
\\* \verb%  printf("Move disk %\verb-%-\verb%d from peg %\verb-%-\verb%d onto peg %\verb-%-\verb%d.\n", %
\\* \verb%         ring_num, start, end);%
\end{flushleft}

\begin{flushleft}
\verb%  return;%
\\* \verb%}%
\end{flushleft}
