%
% CSE 110 Lecture Notes
%
% Entire contents are copyright 1992 by Mark-Jason Dominus.
% All rights reserved.  Unauthorized reproduction prohobited.
%

\documentstyle[fancyheadings]{article}

% \def\brac#1{{\tt <}#1{\tt >}}
\def\brac#1{$<${#1}$>$}
\def\Int{{\tt int}}
\def\int{\brac{\Int}}
\def\int{\brac{\Int}}
\def\Shortint{{\tt short~int}}
\def\shortint{\brac{\Shortint}}
\def\Longint{{\tt long~int}}
\def\longint{\brac{\Longint}}
\def\Float{{\tt float}}
\def\float{\brac{\Float}}
\def\Double{{\tt double}}
\def\double{\brac{\Double}}
\def\Char{{\tt char}}
\def\chr{\brac{\Char}}
\def\Void{{\tt void}}
\def\void{\brac{\Void}}

\def\ptr#1{pointer~to {#1}}
\def\p2#1{\brac{\ptr{#1}}}
\def\Ano#1#2{array~of {#1}~{#2}s}
\def\ano#1#2{\brac{\Ano{#1}{#2}}}
\def\Ao#1{array~of {#1}}
\def\ao#1{\brac{\Ao#1}}

\def\np{{\tt NULL} pointer}

\def\breakhere{\mbox{$\otimes$}}


\title{Lecture 18}
\author{CSE 110}
\date{4 August 1992}

\parskip 8pt

\pagestyle{fancy}
\lhead{CSE 110 Lecture Notes}
\chead{Mark--Jason Dominus}
\rhead{\thepage}
\lfoot{Copyright \copyright 1992 Mark-Jason Dominus}
\cfoot{}
\rfoot{All rights reserved.}



\begin{document}
\maketitle

\section{The Tower of Hanoi}

As promised, here's the code for the Tower of Hanoi program.

In the program, the three pegs are represented internally by the
numbers 1, 2, and 3.  As written, the program reads a number of rings
from the user, and print out instructions to tell the user how to
transfer a tower of this height from ring 1 to ring 3.  

The workhorse function, {\tt hanoi}, moves a tower of {\tt num\_rings}
rings from peg {\tt start\_peg} to peg {\tt end\_peg}.  The first thing
it does is to figure out where the spare peg is; to do this it uses a
simple trick: If pegs $s$ and $e$ are the start and end pegs, then peg
$6-s-e$ is the spare peg.  (For example, if $s$ is 2 and $e$ is 1, the
spare peg is $6-2-1$ or $3$.)

If the height of the tower that {\tt hanoi} is called upon to move has
size 0, it returns immediately, because it doesn't need to do anything.
Otherwise, it calls itself recursively to move the subtower of ${\tt
num\_rings} - 1$ rings from the start peg to the spare beg, calls {\tt
move} to move the largest ring from the start peg to the end peg, and
then calls itself recursively again to move the subtower from the spare
peg to the end peg.

\begin{flushleft}
\verb%void%
\\* \verb%  hanoi(int num_rings,%
\\* \verb%        int start_peg, %
\\* \verb%        int end_peg)%
\\* \verb%{%
\\* \verb%  int spare_peg = 6 - start_peg - end_peg;%

\verb%  if (num_rings > 0) {%
\\* \verb%    hanoi(num_rings - 1, start_peg, spare_peg);%
\\* \verb%    move(num_rings, start_peg, end_peg);%
\\* \verb%    hanoi(num_rings - 1, spare_peg, end_peg);%
\\* \verb%  }%

\verb%  return;%
\\* \verb%}%
\end{flushleft}

{\tt move} is the function which is called each time we want to move a
particular single ring.  If we were writing our Tower of Hanoi program
to do a fancy screen display which showed the rings flying around from
peg to peg, we would put the code for drawing the rings on the screen in
{\tt move}.  In this simple program, we'll be content to just print out
an instruction about what peg should be moved where:

\begin{flushleft}
\verb%void move(int ring_num, int start, int end)%
\\* \verb%{%
\\* \verb%  printf("Move disk %\verb-%-\verb%d from peg %\verb-%-\verb%d onto peg %\verb-%-\verb%d.\n", %
\\* \verb%         ring_num, start, end);%

\verb%  return;%
\\* \verb%}%
\end{flushleft}

We'll add a {\tt main} which examines its command-line arguments
to decide how many rings the user wants, and then just calls {\tt hanoi}
to print the instructions for moving a tower of that many rings from peg
1 to peg 3:

\begin{flushleft}
\verb%void hanoi(int num_rings, int start_peg, int end_peg);%
\\* \verb%void move(int ring_num, int start, int end);%
\end{flushleft}

\begin{flushleft}
\verb%int main(int argc, char **argv)%
\\* \verb%{%
\\* \verb%  int num_rings;%

\verb%  if (argc != 2) {%
\\* \verb%    fprintf(stderr, "Usage: %\verb-%-\verb%s number_of_rings\n", argv[0]);%
\\* \verb%    return 1;%
\\* \verb%  }%

\verb%  num_rings = atoi(argv[1]);%

\verb%  hanoi(num_rings, 1, 3);%

\verb%  return 0;%
\\* \verb%}%
\end{flushleft}

    The {\tt atoi} function is something new: It accepts a string which
is supposed to contain the string representation of an integer, and it
returns the integer that the string represents.  For example {\tt
atoi("119")} returns the integer 119.  {\tt atoi} returns 0 if there is
an error, for example, because the string passed in was not composed of
digits.

The program is almost trivial with recursion, but it would be very
difficult to do without recursion, because the solution we have, to
reduce the problem to a simpler case, is already organized along
recursive lines.

\end{document}
