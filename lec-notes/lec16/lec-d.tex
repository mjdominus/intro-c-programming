
\documentstyle{article}

% \def\brac#1{{\tt <}#1{\tt >}}
\def\brac#1{$<$#1$>$}
\def\Int{{\tt int}}
\def\int{\brac{\Int}}
\def\int{\brac{\Int}}
\def\Shortint{{\tt short~int}}
\def\shortint{\brac{\Shortint}}
\def\Longint{{\tt long~int}}
\def\longint{\brac{\Longint}}
\def\Float{{\tt float}}
\def\float{\brac{\Float}}
\def\Double{{\tt double}}
\def\double{\brac{\Double}}
\def\Char{{\tt char}}
\def\chr{\brac{\Char}}
\def\Void{{\tt void}}
\def\void{\brac{\Void}}

\def\ptr#1{pointer~to #1}
\def\p2#1{\brac{\ptr#1}}
\def\Ano#1#2{array~of {#1}~#2s}
\def\ano#1#2{\brac{\Ano#1#2}}
\def\Ao#1{array~of #1}
\def\ao#1{\brac{\Ao#1}}

\def\np{{\tt NULL} pointer}

\def\breakhere{\mbox{$\otimes$}}


\title{Lecture 15}
\author{CSE 110}
\date{25 July 1992}

\parskip 8pt

\begin{document}
\maketitle

\section{Dynamic Memory Allocation}

We are finally going to learn how to allocate memory on the fly, getting
more when we need it, a subject not usually covered in introductory C
courses.

\subsection{{\tt malloc}}

We've already seen {\tt strdup}, which find memory somewhere and reserves
it.  How did {\tt strdup} find that memory?

Chances are it called {\tt malloc}.  {\tt malloc} finds memory by asking
the operating system for a big bunch of memory and then parceling it
out a little at a time as you ask for it.  It's {\tt malloc} that takes
care of recording how big each parcel is so that {\tt free} can free it
properly.  

To call {\tt malloc}, pass as an argument the number of bytes of space
you want to allocate.  {\tt malloc} will reserve that many bytes and
return a pointer to the memory it found.  If it fails for any reason,
such as because all the memory is already reserved, {\tt malloc} returns
(all together now) the {\tt NULL} pointer.  

\subsection{{\tt sizeof}}

Frequently you want to allocate enough memory to store an object of a
certain type; you need to be able to tell {\tt malloc} how many bytes of
space the object takes up so it knows how much to allocate.  There's an
operator in C which tells you how big an object is: {\tt sizeof}. 

If {\em type} is the name of a type, the value of the expression {\tt
sizeof({\em type}\/)} is the number of bytes needed to hold that type;
for example, on our machines, {\tt sizeof(int)} is 2 and {\tt
sizeof(double)} is 8.  

If we want to get enough space for an array of 13 \int s, we can do {\tt
malloc(13 * sizeof(int))} or even {\tt malloc(sizeof(int[13]))}.    

\section{{\tt struct}s}

We've seen how to use an array to store many values of the same type,
and even to treat the collection of values as a unit.  Now we'll look at
the complementary type, the {\em structure}\/, which allows you to group
several different object together as a unit.

\subsection{Creating a New Structure Type}

The typical example of a structure is this:  You want to keep track of
the employees in your organization.  Each employee has a name (a string
of no more than 50 characters), a social security number (a \longint),
and a salary (an \int).  You want to handle several such records.

You can create a new type, a {\tt struct employee}, which keeps all of
this information in one place.  First, you define the new type:

\begin{flushleft}
\verb% struct employee {% \\*
\verb%   char name[50];% \\*
\verb%   long int ssn;% \\*
\verb%   int salary;% \\*
\verb%} ; % 
\end{flushleft}

When you write this in your program, it defines a new type, the type
{\tt struct employee}.  A variable of type {\tt struct employee} has
three {\em members}\/:  {\tt name}, an array of 50 \chr s, {\tt ssn}, a
\longint, and {\tt salary}, an \int.  
{\tt employee} is called the {\em tag}\/ of the structure.

\subsection{Creating {\tt struct} Variables}

To declare a variables of type {\tt struct employee}, just write:

\begin{flushleft}
\verb% struct employee smithers, marketing[12], *the_employee;% 
\end{flushleft}

{\tt smithers} is a {\tt struct employee}; {\tt marketing} is an array
of twelve of these structures, and {\tt the\_employee} is a pointer to
one of these structures.

\subsection{The {\tt .} Operator}

Suppose {\tt smithers} is a {\tt struct employee} variable, as above.
Suppose we want to store the information that Smithers' social security
number is {\tt 314-15-9265}.  Here's how we might do that:

\begin{flushleft}
\verb% smithers.ssn = 314159265; %
\end{flushleft}

The {\tt .} operator is the {\em structure member}\/ operator.  Its left
operand must be a structure of some type; its right operand must be the
name of one of the members of that structure.  The entire expression
refers to the specified member of the specified structure.  So
similarly, 

\begin{flushleft}
\verb% if ( smithers.salary > 10000 ) { %  \\*
\verb%   ... % 
\end{flushleft}

\noindent asks if Smithers' salary exceeds ten thousand dollars. 

Similarly, let's suppose the array {\tt marketing}, which is an array of
12 {\tt struct employee}s, has been initialized with the names,
salaries, and social security numbers of the members of the marketing
department.  Here's how we'd compute the average salary in the marketing
department:  

\begin{flushleft}
\verb% long int totalsalary = 0;% \\*
\verb% % \\*
\verb% for (i=0; i<12; i++) % \\*
\verb%   totalsalary += marketing[i].salary;% \\*
\verb! printf("Average salary is %f.\n", totalsalary/12.0);! \\*
\end{flushleft}

{\tt marketing} is an array of {\tt struct employee}s, so {\tt
marketing[i]} is a {\tt struct employee}, and {\tt marketing[i].salary}
is the salary member of that {\tt struct employee}. 

\section{The Linked List}

The simplest example of a data structure with no fixed size is the {\em
linked list}\/.  A linked list is a collection of {\em nodes} in some
order; each node contains some information about how to find the `next'
node, and perhaps some auxiliary information as well.  The last node has
some kind of marker that says it's last.  

To keep track of a linked list, all we need to do is remember where the
first node is.  Then that first node contains information (called a {\em
link}\/ that tells us where the second node is; the second node contains
information about where the third node is, and soforth. Clearly the list
can be any length.  

Here's a simple way to implement a linked list: it keeps track of as
many numbers as we like:

\begin{flushleft}
\verb% struct listnode {% \\*
\verb%   int data; % \\*
\verb%   struct listnode *next; % \\*
\verb%} ; % 
\end{flushleft}

This {\tt struct} has two members:  {\tt data}, which actually holds a
number, and {\tt next}, which points to the next node in the list.  

Suppose the location of the first node in the list is stored in the
variable {\tt firstnode}, whose type is {\tt struct listnode *}.
Suppose also we discover that we need to remember one more number, {\tt
newnumber}.  How can we add that number onto the list of numbers in a
list of these nodes?

\begin{flushleft}
\verb% struct listnode *temp;% \\* 
\verb% % \\* 
\verb% temp = malloc(sizeof(struct listnode));% \\* 
\verb% % \\* 
\verb% (*temp).next = firstnode;% \\* 
\verb% (*temp).data = newnumber;% \\* 
\verb% firstnode = temp;% \\* 
\end{flushleft}

First we create enough space for a new list node; we link the node to
the rest of the list, store the number we want to remember into the data
element, and then remember that the new node is now first, with {\tt
firstnode = temp}.  We can do this as many times as we want, until the
memory runs out.  

Now suppose we want to average the numbers stored in each node.  It's
easy:

\begin{flushleft}
\verb% struct listnode *current;%  \\*
\verb% int average;%  \\*
\verb% long int sum = 0L;%  \\*
\verb% int nodecount = 0;%  \\*
\verb% %  \\*
\verb% for (listnode = firstnode; % \*
\verb%      listnode != NULL; %  \\*
\verb%      listnode = (*listnode).next) {%  \\*
\verb%   sum += (*listnode).data; %  \\*
\verb%   nodecount += 1;%  \\*
\verb% }%  \\*
\verb% %  \\*
\verb% average = sum / nodecount;%  \\*
\end{flushleft}

Here we've assumed that the {\tt next} pointer of the last node in the
list is {\tt NULL}.  This makes perfect sense, and we can be sure that
no other node but the last has {\tt NULL} for its {\t next} pointer,
because all the other {\tt next} pointers point to other nodes, whose
addresses are not {\tt NULL}. 

I hope it's obvious how this is all relevant to the problem of managing
an arbitrarily large stack for your calculator.  

\subsection{The {\tt ->} Operator}

The operation of accessing the member of a structure that is pointed to
by a structure pointer variable is common.  We've used the construction
{\tt (*foo).bar} several times already.  C lets you use a short cut.

In all circumstances, the expression {\tt foo->bar} is identical with
{\tt (*foo).bar}.  {\tt foo} must be a pointer to a structure, and {\tt
bar} must be the name of one of the members of that structure.  

\end{document}


