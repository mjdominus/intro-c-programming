
\documentstyle{article}

% \def\brac#1{{\tt <}#1{\tt >}}
\def\brac#1{$<$#1$>$}
\def\Int{{\tt int}}
\def\int{\brac{\Int}}
\def\int{\brac{\Int}}
\def\Shortint{{\tt short~int}}
\def\shortint{\brac{\Shortint}}
\def\Longint{{\tt long~int}}
\def\longint{\brac{\Longint}}
\def\Float{{\tt float}}
\def\float{\brac{\Float}}
\def\Double{{\tt double}}
\def\double{\brac{\Double}}
\def\Char{{\tt char}}
\def\chr{\brac{\Char}}
\def\Void{{\tt void}}
\def\void{\brac{\Void}}

\def\ptr#1{pointer~to #1}
\def\p2#1{\brac{\ptr#1}}
\def\Ano#1#2{array~of {#1}~#2s}
\def\ano#1#2{\brac{\Ano#1#2}}
\def\Ao#1{array~of #1}
\def\ao#1{\brac{\Ao#1}}

\def\np{{\tt NULL} pointer}

\def\breakhere{\mbox{$\otimes$}}

\title{Assignment \#3}
\author{Due Date: Thursday, 6 August, 1992}
\date{15 July 1992}

\parskip 8pt

\begin{document}
\maketitle

    You will write a program which reads arithmetic expressions,
evaluates them, and prints the result.

\section{Requirements}

Your program will implement a reverse Polish notation calculator.  The
program will read from the standard input.  Your program will compute
the values of all expressions entered.  Expressions will consist of
integers and operators.  Your program will recognize the operators {\tt
+}, {\tt -}, {\tt *}, {\tt /}, and {\tt \%}.  Furthermore, your program
will print out the most recently computed result when it sees the
operator {\tt =}.  Your program will discard the most recently computed
result when it sees the operator {\tt .}.  

Your program will not place any arbitrary limits on the complexity of
the expressions the user can evaluate.  Your program will handle error
conditions as gracefully as possible and will print appropriate error
messages if memory is exhausted or if the expression the user enters
contains an error.

\subsection{Example}

Indented lines represent user input; other lines represent output.
\begin{flushleft}
\verb!         2 3 * =! \\*
\verb! 6! \\*
\verb!         4 + = ! \\*
\verb! 10 ! \\*
\verb!         7 - = ! \\*
\verb! 3 ! \\*
\verb!         5 6 * = + = ! \\*
\verb! 30! \\*
\verb! 33! \\*
\verb!         12 3 * = ! \\*
\verb! 36 ! \\*
\verb!         .! \\*
\verb!         =! \\*
\verb! 33 ! \\*
\verb!         12 12 12 * * 1 + = ! \\*
\verb! 1729 ! \\*
\verb!         . =  ! \\*
\verb! 33 ! \\*
\verb!         .! \\*
\verb!         =! \\*
\verb! (stack empty)! \\*
\end{flushleft}

\section{What to Hand in}

You should hand in a disk with your source code and an executable, and
any special instructions to me.  You may choose to supply a log file
that demonstrates your program, if you like.  I will not accept paper
copies of solutions to this assignment.

\section{Assignment-Specific Points}


\section{About Reverse Polish Notation}
\label{rpn}

Evaluating ordinary arithmetic expressions is a pain, because you might
have to remember arbitrarily large expressions for a long time until you
can evaluate them.  For example, If the input stream contains {\tt 2 +
3} you can't forget the {\tt 2 + 3} and replace it with 5 immediately,
because it might be the first part of {\tt 2 + 3 * 4}.  Also, it's
unreasonable to write a calculator program that evaluates expressions of
this type without having it handle parentheses, and that's also painful.

Fortunately there is a solution, adopted by programming language
designers, calculator manufacturers, logicians, and us.  We require that
our expressions be entered in {\em Reverse Polish Notation}\/.  What
this means is that instead of each operator coming between the two
quantities it's supposed to operate on (which is ambiguous), each
operator follows the two quantities it's supposed to operate on.

A reverse Polish notation expressions is either just a number, or two
simpler expressions followed by an operator symbol, which says how to
combine the values of the two expressions. 
For example:  
\begin{itemize}
\item The value of {\tt 2 3 +}  is 5.
\item The value of {\tt 3 4 *}  is 12.
\item The value of {\tt 2 3 + 3 4 * +} is 17; the final {\tt +} says to
evaluate the two expressions that came before and add their values
together.
\item The value of {\tt 2 3 + 4 *} is 20; the final l{\tt *} says to
take the values of the two preceding expressions, {\tt 2 3 +} and {\tt
4}, and multiply them.
\item The value of {\tt 2 3 4 * +} is 14; the final {\tt +} says to take
the values of the two preceding expressions, {\tt 2} and {\tt 3 4 *},
and add them.
\end{itemize}

Issues of precedence and parenthesization disappear, because it's no
longer ambiguous where a subexpression ends, and therefore it's clear
what the operands of each operator symbol are.  Every expression and
sub-expression ends with an operator symbol.  

It turns out that interpreting and evaluating these expressions is very
easy.  

\section{Stacks}

A {\em stack}\/ is like an array, but only one element is accessible at
any time:  The {\em top}\/ element. There are just two things you can do
with a stack:  You can {\em push}\/ a value onto the stack, in which
case that value becomes the top value, and you can {\em pop}\/ the top
value off of the stack and examine it, in which case the element that
was pushed just before it becomes the new top.

The name `stack' is supposed to make you think of those stacks of plates
with springs underneath that you sometimes find in cafeterias.  The
spring is under a whole stack of plates, but it keeps them at the right
height so that only the top plate is visible.  You can only see the top
plate; you can only remove the top plate; but when you take the top
plate off, the plate just below it appears and then you can examine or
remove that plate.  When you put a new plate on top, it obscures the
plate that used to be visible.

It turns out that a stack is just  the thing for implementing an RPN
calculator.  The rules are simple:  Read the input from right to left. 
\begin{enumerate}
\item If you see a number, push it onto the stack.
\item If you see an operator, pop the right number of operands off the
stack, operate on them, and push the result on the stack.
\end{enumerate}

For example, let's say the user enters the expression {\tt 2 3 * 4 +}.
We should compute the value 10.  What do we do?  First, we see {\tt 2},
so we push that on the stack; then we push the {\tt 3} after it.  Then
we see the {\tt *}, so we pop the {\tt 2} and the {\tt 3} from the
stack, multiply them, and push the result, 6.  Then we push the {\tt 4};
the stack now contains a {\tt 6} and a {\tt 4} with the {\tt 4} at the
top.    Then we see the {\tt +}; we pop the {\tt 6} and the {\tt 4}, add
them, and push the result, 10, back on the stack.  The top of the stack
now contains the result {\tt 10}, which was the value we wanted to
compute.

Let's do another example:  {\tt 2 3 4 + *}.  We see the {\tt 2}, the
{\tt 3}, and the {\tt 4}, and push them on the stack in that order, so
that the {\tt 4} is at the top and the {\tt 2} is at the bottom.  Then
we see the {\tt +}; we pop the top two elements (The {\tt 3} and the
{\tt 4} off the stack, add them, and push the result, 7, back on the
stack.  The stack now contains two elements, {\tt 2} and {\tt 7}, with
the {\tt 7} on top.  Then we see the {\tt *}, pop the two elements off
the stack, multiply them, and push the result, 14, back on.  The top [of
the stack now contains the value 14, which is what we wanted to compute.

\subsection{A Simple Implementation of a Stack}

\input ss.tex

Unfortunately, this is a little too simple.  This program terminates if
the user tries to make the stack too deep by computing too many
intermediate results.  Your program will not have this problem.  We will
arrange for our stack to grow as large as necessary, until the computer
runs out of memory.

\section{Dynamic Memory Allocation}

We are finally going to learn how to allocate memory on the fly, getting
more when we need it, a subject not usually covered in introductory C
courses.

\subsection{{\tt malloc}}

We've already seen {\tt strdup}, which find memory somewhere and reserves
it.  How did {\tt strdup} find that memory?

Chances are it called {\tt malloc}.  {\tt malloc} finds memory by asking
the operating system for a big bunch of memory and then parceling it
out a little at a time as you ask for it.  It's {\tt malloc} that takes
care of recording how big each parcel is so that {\tt free} can free it
properly.  

To call {\tt malloc}, pass as an argument the number of bytes of space
you want to allocate.  {\tt malloc} will reserve that many bytes and
return a pointer to the memory it found.  If it fails for any reason,
such as because all the memory is already reserved, {\tt malloc} returns
(all together now) the {\tt NULL} pointer.  

\subsection{{\tt sizeof}}

Frequently you want to allocate enough memory to store an object of a
certain type; you need to be able to tell {\tt malloc} how many bytes of
space the object takes up so it knows how much to allocate.  There's an
operator in C which tells you how big an object is: {\tt sizeof}. 

If {\em type} is the name of a type, the value of the expression {\tt
sizeof({\em type}\/)} is the number of bytes needed to hold that type;
for example, on our machines, {\tt sizeof(int)} is 2 and {\tt
sizeof(double)} is 8.  

If we want to get enough space for an array of 13 \int s, we can do {\tt
malloc(13 * sizeof(int))} or even {\tt malloc(sizeof(int[13]))}.    

\section{{\tt struct}s}

We've seen how to use an array to store many values of the same type,
and even to treat the collection of values as a unit.  Now we'll look at
the complementary type, the {\em structure}\/, which allows you to group
several different object together as a unit.

\subsection{Creating a New Structure Type}

The typical example of a structure is this:  You want to keep track of
the employees in your organization.  Each employee has a name (a string
of no more than 50 characters), a social security number (a \longint),
and a salary (an \int).  You want to handle several such records.

You can create a new type, a {\tt struct employee}, which keeps all of
this information in one place.  First, you define the new type:

\begin{flushleft}
\verb% struct employee {% \\*
\verb%   char name[50];% \\*
\verb%   long int ssn;% \\*
\verb%   int salary;% \\*
\verb%} ; % 
\end{flushleft}

When you write this in your program, it defines a new type, the type
{\tt struct employee}.  A variable of type {\tt struct employee} has
three {\em members}\/:  {\tt name}, an array of 50 \chr s, {\tt ssn}, a
\longint, and {\tt salary}, an \int.  
{\tt employee} is called the {\em tag}\/ of the structure.

\subsection{Creating {\tt struct} Variables}

To declare a variables of type {\tt struct employee}, just write:

\begin{flushleft}
\verb% struct employee smithers, marketing[12], *the_employee;% 
\end{flushleft}

{\tt smithers} is a {\tt struct employee}; {\tt marketing} is an array
of twelve of these structures, and {\tt the\_employee} is a pointer to
one of these structures.

\subsection{The {\tt .} Operator}

Suppose {\tt smithers} is a {\tt struct employee} variable, as above.
Suppose we want to store the information that Smithers' social security
number is {\tt 314-15-9265}.  Here's how we might do that:

\begin{flushleft}
\verb% smithers.ssn = 314159265; %
\end{flushleft}

The {\tt .} operator is the {\em structure member}\/ operator.  Its left
operand must be a structure of some type; its right operand must be the
name of one of the members of that structure.  The entire expression
refers to the specified member of the specified structure.  So
similarly, 

\begin{flushleft}
\verb% if ( smithers.salary > 10000 ) { %  \\*
\verb%   ... % 
\end{flushleft}

\noindent asks if Smithers' salary exceeds ten thousand dollars. 

Similarly, let's suppose the array {\tt marketing}, which is an array of
12 {\tt struct employee}s, has been initialized with the names,
salaries, and social security numbers of the members of the marketing
department.  Here's how we'd compute the average salary in the marketing
department:  

\begin{flushleft}
\verb% long int totalsalary = 0;% \\*
\verb% % \\*
\verb% for (i=0; i<12; i++) % \\*
\verb%   totalsalary += marketing[i].salary;% \\*
\verb! printf("Average salary is %f.\n", totalsalary/12.0);! \\*
\end{flushleft}

{\tt marketing} is an array of {\tt struct employee}s, so {\tt
marketing[i]} is a {\tt struct employee}, and {\tt marketing[i].salary}
is the salary member of that {\tt struct employee}. 

\section{The Linked List}

The simplest example of a data structure with no fixed size is the {\em
linked list}\/.  A linked list is a collection of {\em nodes} in some
order; each node contains some information about how to find the `next'
node, and perhaps some auxiliary information as well.  The last node has
some kind of marker that says it's last.  

To keep track of a linked list, all we need to do is remember where the
first node is.  Then that first node contains information (called a {\em
link}\/ that tells us where the second node is; the second node contains
information about where the third node is, and soforth. Clearly the list
can be any length.  

Here's a simple way to implement a linked list: it keeps track of as
many numbers as we like:

\begin{flushleft}
\verb% struct listnode {% \\*
\verb%   int data; % \\*
\verb%   struct listnode *next; % \\*
\verb%} ; % 
\end{flushleft}

This {\tt struct} has two members:  {\tt data}, which actually holds a
number, and {\tt next}, which points to the next node in the list.  

Suppose the location of the first node in the list is stored in the
variable {\tt firstnode}, whose type is {\tt struct listnode *}.
Suppose also we discover that we need to remember one more number, {\tt
newnumber}.  How can we add that number onto the list of numbers in a
list of these nodes?

\begin{flushleft}
\verb% struct listnode *temp;% \\* 
\verb% % \\* 
\verb% temp = malloc(sizeof(struct listnode));% \\* 
\verb% % \\* 
\verb% (*temp).next = firstnode;% \\* 
\verb% (*temp).data = newnumber;% \\* 
\verb% firstnode = temp;% \\* 
\end{flushleft}

First we create enough space for a new list node; we link the node to
the rest of the list, store the number we want to remember into the data
element, and then remember that the new node is now first, with {\tt
firstnode = temp}.  We can do this as many times as we want, until the
memory runs out.  

Now suppose we want to average the numbers stored in each node.  It's
easy:

\begin{flushleft}
\verb% struct listnode *current;%  \\*
\verb% int average;%  \\*
\verb% long int sum = 0L;%  \\*
\verb% int nodecount = 0;%  \\*
\verb% %  \\*
\verb% for (listnode = firstnode; % \*
\verb%      listnode != NULL; %  \\*
\verb%      listnode = (*listnode).next) {%  \\*
\verb%   sum += (*listnode).data; %  \\*
\verb%   nodecount += 1;%  \\*
\verb% }%  \\*
\verb% %  \\*
\verb% average = sum / nodecount;%  \\*
\end{flushleft}

Here we've assumed that the {\tt next} pointer of the last node in the
list is {\tt NULL}.  This makes perfect sense, and we can be sure that
no other node but the last has {\tt NULL} for its {\t next} pointer,
because all the other {\tt next} pointers point to other nodes, whose
addresses are not {\tt NULL}. 

I hope it's obvious how this is all relevant to the problem of managing
an arbitrarily large stack for your calculator.  

\subsection{The {\tt ->} Operator}

The operation of accessing the member of a structure that is pointed to
by a structure pointer variable is common.  We've used the construction
{\tt (*foo).bar} several times already.  C lets you use a short cut.

In all circumstances, the expression {\tt foo->bar} is identical with
{\tt (*foo).bar}.  {\tt foo} must be a pointer to a structure, and {\tt
bar} must be the name of one of the members of that structure.  

\end{document}


