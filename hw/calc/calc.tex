
\documentstyle{article}

% \def\brac#1{{\tt <}#1{\tt >}}
\def\brac#1{$<$#1$>$}
\def\Int{{\tt int}}
\def\int{\brac{\Int}}
\def\int{\brac{\Int}}
\def\Shortint{{\tt short~int}}
\def\shortint{\brac{\Shortint}}
\def\Longint{{\tt long~int}}
\def\longint{\brac{\Longint}}
\def\Float{{\tt float}}
\def\float{\brac{\Float}}
\def\Double{{\tt double}}
\def\double{\brac{\Double}}
\def\Char{{\tt char}}
\def\chr{\brac{\Char}}
\def\Void{{\tt void}}
\def\void{\brac{\Void}}

\def\ptr#1{pointer~to #1}
\def\p2#1{\brac{\ptr#1}}
\def\Ano#1#2{array~of {#1}~#2s}
\def\ano#1#2{\brac{\Ano#1#2}}
\def\Ao#1{array~of #1}
\def\ao#1{\brac{\Ao#1}}

\def\np{{\tt NULL} pointer}

\def\breakhere{\mbox{$\otimes$}}

\title{Assignment \#3}
\author{Due Date: Thursday, 6 August, 1992}
\date{28 July 1992}

\parskip 8pt

\begin{document}
\maketitle

    You will write a program which reads arithmetic expressions,
evaluates them, and prints the result.

\section{Requirements}

Your program will implement a reverse Polish notation calculator.  The
program will read from the standard input.  Your program will compute
the values of all expressions entered.  Expressions will consist of
integers and operators.  Your program will recognize the operators {\tt
+}, {\tt -}, {\tt *}, {\tt /}, and {\tt \%}.  Furthermore, your program
will print out the most recently computed result when it sees the
operator {\tt =}.  Your program will discard the most recently computed
result when it sees the operator {\tt .}.  

Your program will not place any arbitrary limits on the complexity of
the expressions the user can evaluate.  Your program will handle error
conditions as gracefully as possible and will print appropriate error
messages if memory is exhausted or if the expression the user enters
contains an error.

\subsection{Example}

Indented lines represent user input; other lines represent output.
\begin{flushleft}
\verb!         2 3 * ! \\*
\verb!         =! \\*
\verb! 6! \\*
\verb!         4 + = ! \\*
\verb! 10 ! \\*
\verb!         7 - = ! \\*
\verb! 3 ! \\*
\verb!         5 6 * = + = ! \\*
\verb! 30! \\*
\verb! 33! \\*
\verb!         12 3 * = ! \\*
\verb! 36 ! \\*
\verb!         .! \\*
\verb!         =! \\*
\verb! 33 ! \\*
\verb!         12 12 12 * * 1 + = ! \\*
\verb! 1729 ! \\*
\verb!         . =  ! \\*
\verb! 33 ! \\*
\verb!         .! \\*
\verb!         =! \\*
\verb! (stack empty)! \\*
\end{flushleft}

\section{What to Hand in}

You should hand in a disk with your source code and an executable, and
any special instructions to me.  You may choose to supply a log file
that demonstrates your program, if you like.  I will not accept paper
copies of solutions to this assignment.

\section{Assignment-Specific Points}

    I will award seven points for a working RPN calculator, and seven
points for arranging that the calculator can handle arbitrarily
complicated expressions.

\section{About Reverse Polish Notation}
\label{rpn}

Evaluating ordinary arithmetic expressions is a pain, because you don't
get to evaluate things in the order they're entered.  To evaluate the
expression \mbox{\tt 4 * ( 3 + 7 - 6 * 2) + 11} you have to skip forward to
the part in the parentheses, then skip back; you can't evaluate it from
left to right.

Fortunately there is an easier way, adopted by programming language
designers, calculator manufacturers, logicians, and us.  We will require
that our expressions be entered in {\em reverse Polish notation}\/.
What this means is that instead of each operator coming between the two
quantities it's supposed to operate on (which is ambiguous), each
operator follows the two quantities it's supposed to operate on.


A reverse Polish notation expression, therefore, is either just a
number, or two simpler expressions {\em followed}\/ by an operator
symbol, which says how to combine the values of the two expressions.
For example:
\begin{itemize}
\item The value of {\tt 2 3 +}  is 5.
\item The value of {\tt 3 4 *}  is 12.
\item The value of {\tt 2 3 + 3 4 * +} is 17; the final {\tt +} says to
evaluate the two expressions ({\tt 2 3 +} and {\tt 3 4 *}) that came
before and add their values together.
\item The value of {\tt 2 3 + 4 *} is 20; the final {\tt *} says to
take the values of the two preceding expressions, {\tt 2 3 +} and {\tt
4}, and multiply them.
\item The value of {\tt 2 3 4 * +} is 14; the final {\tt +} says to take
the values of the two preceding expressions, {\tt 2} and {\tt 3 4 *},
and add them.
\end{itemize}

Issues of precedence and parenthesization disappear, because it's no
longer ambiguous where a subexpression ends, and therefore it's clear
what the operands of each operator symbol are.  Every expression and
sub-expression ends with an operator symbol.  

Reverse polish notation is great for calculators because you get to
write the operators in the order that the operations are actually
performed.  If you heed to compute two quantities, $x$ and $y$, and add
them together, then you first compute $x$, then compute $y$, and then do
the addition---of course it's absurd to say you want to do the addition
before you've computed $y$; that's impossible.  Reverse Polish notation
lets you actually write the expressions in the order they're computed:
\mbox{$x$ $y$ +} means `compute $x$; then compute $y$; then add those
two values.'

Since the operations in an RPN expression are written in the order that
they actually have to be performed, it turns out that interpreting and
evaluating these expressions is very easy.  There's a natural way to
implement an RPN calculator, which is the subject of the next section.

\section{Stacks}

The name {\em stack}\/ is supposed to make you think of those stacks of
plates with springs underneath that you sometimes find in cafeterias.
The spring is under a whole stack of plates, but it keeps them at the
right height so that only the top plate is visible.  You can only see
the top plate; you can only remove the top plate; but when you take the
top plate off, the plate just below it appears and then you can examine
or remove that plate.  When you put a new plate on top, it obscures the
plate that used to be visible.

So a stack is a data structure that has two operations defined on it:
You can {\em pop}\/ the stack; that means you take the top element off
the stack and examine it; the element under the top element becomes the
new top.  We also speak of popping the top element itself.  You can also
{\em push}\/ a new element onto the stack.  If you push {\tt foo} onto a
stack and then push {\tt bar}, {\tt bar} is on the top of the stack and
{\tt foo} is under that; if you pop the stack you get {\tt bar}, and
then if you pop it again you get {\tt foo}.  The only way to examine the
data at the bottom of the stack is to pop off everything else above it.
If you try to pop the stack when the stack is empty, you get an error.

It turns out that a stack is just  the thing for implementing an RPN
calculator.  The rules are simple: 
\begin{enumerate}
\item Read the input from left to right. 
\item If you see a number, push it onto the stack.
\item If you see an operator, pop the right number of operands off the
stack, operate on them, and push the result on the stack.
\end{enumerate}

For example, let's say the user enters the expression {\tt 2 3 * 4 +}.
We should compute the value 10.  What do we do?  First, we see {\tt 2},
so we push that on the stack; then we push the {\tt 3} after it.  Then
we see the {\tt *}, so we pop the {\tt 3} and the {\tt 2} from the
stack, multiply them, and push the result, 6.  Then we push the {\tt 4};
the stack now contains a {\tt 6} and a {\tt 4} with the {\tt 4} at the
top.  Then we see the {\tt +}; we pop the {\tt 6} and the {\tt 4}, add
them, and push the result, 10, back on the stack.  The top of the stack
now contains the result {\tt 10}, which was the value we wanted to
compute.

Let's do another example:  {\tt 2 3 4 + *}.  We see the {\tt 2}, the
{\tt 3}, and the {\tt 4}, and push them on the stack in that order, so
that the {\tt 4} is at the top and the {\tt 2} is at the bottom.  Then
we see the {\tt +}; we pop the top two elements (The {\tt 4} and the
{\tt 3} off the stack, add them, and push the result, 7, back on the
stack.  The stack now contains two elements, {\tt 2} and {\tt 7}, with
the {\tt 7} on top.  Then we see the {\tt *}, pop the two elements off
the stack, multiply them, and push the result, 14, back on.  The top [of
the stack now contains the value 14, which is what we wanted to compute.

\subsection{A Simple Implementation of a Stack}

Here's code that implements a stack of \int s and two functions for
pushing and popping.

\input ss.tex

Unfortunately, this is a little too simple.  This program terminates if
the user tries to make the stack too deep by computing too many
intermediate results.  Your program will not have this problem.  We will
arrange for our stack to grow as large as necessary, until the computer
runs out of memory.  Probably the simplest way to do this is to
implement your stack as a linked list.

If we don't discuss linked lists in class by the end of the 29th of
July, please mention it to me and I'll hand out detailed information
about them on the 30th.  

\end{document}


